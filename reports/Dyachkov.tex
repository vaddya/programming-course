\documentclass[12pt,a4paper]{report}
\usepackage{ucs} 
\usepackage[utf8]{inputenc}
\usepackage[russian]{babel}
\usepackage[OT1]{fontenc}
\usepackage{amsmath}
\usepackage{amsfonts}
\usepackage{amssymb}
\usepackage{graphicx}
\usepackage{cmap}					% поиск в PDF
\usepackage{mathtext} 				% русские буквы в формулах
%\usepackage{tikz-uml}               % uml диаграммы

% TODOs
\usepackage[%
  colorinlistoftodos,
  shadow
]{todonotes}

% Генератор текста
\usepackage{blindtext}

%------------------------------------------------------------------------------

% Подсветка синтаксиса
\usepackage{color}
\usepackage{xcolor}
\usepackage{listings}
 
 % Цвета для кода
\definecolor{string}{HTML}{B40000} % цвет строк в коде
\definecolor{comment}{HTML}{008000} % цвет комментариев в коде
\definecolor{keyword}{HTML}{1A00FF} % цвет ключевых слов в коде
\definecolor{morecomment}{HTML}{8000FF} % цвет include и других элементов в коде
\definecolor{captiontext}{HTML}{FFFFFF} % цвет текста заголовка в коде
\definecolor{captionbk}{HTML}{999999} % цвет фона заголовка в коде
\definecolor{bk}{HTML}{FFFFFF} % цвет фона в коде
\definecolor{frame}{HTML}{999999} % цвет рамки в коде
\definecolor{brackets}{HTML}{B40000} % цвет скобок в коде
 
 % Настройки отображения кода
\lstset{
language=C, % Язык кода по умолчанию
morekeywords={*,...}, % если хотите добавить ключевые слова, то добавляйте
 % Цвета
keywordstyle=\color{keyword}\ttfamily\bfseries,
stringstyle=\color{string}\ttfamily,
commentstyle=\color{comment}\ttfamily\itshape,
morecomment=[l][\color{morecomment}]{\#}, 
 % Настройки отображения     
breaklines=true, % Перенос длинных строк
basicstyle=\ttfamily\footnotesize, % Шрифт для отображения кода
backgroundcolor=\color{bk}, % Цвет фона кода
%frame=lrb,xleftmargin=\fboxsep,xrightmargin=-\fboxsep, % Рамка, подогнанная к заголовку
frame=tblr
rulecolor=\color{frame}, % Цвет рамки
tabsize=3, % Размер табуляции в пробелах
showstringspaces=false,
 % Настройка отображения номеров строк. Если не нужно, то удалите весь блок
numbers=left, % Слева отображаются номера строк
stepnumber=1, % Каждую строку нумеровать
numbersep=5pt, % Отступ от кода 
numberstyle=\small\color{black}, % Стиль написания номеров строк
 % Для отображения русского языка
extendedchars=true,
literate={Ö}{{\"O}}1
  {Ä}{{\"A}}1
  {Ü}{{\"U}}1
  {ß}{{\ss}}1
  {ü}{{\"u}}1
  {ä}{{\"a}}1
  {ö}{{\"o}}1
  {~}{{\textasciitilde}}1
  {а}{{\selectfont\char224}}1
  {б}{{\selectfont\char225}}1
  {в}{{\selectfont\char226}}1
  {г}{{\selectfont\char227}}1
  {д}{{\selectfont\char228}}1
  {е}{{\selectfont\char229}}1
  {ё}{{\"e}}1
  {ж}{{\selectfont\char230}}1
  {з}{{\selectfont\char231}}1
  {и}{{\selectfont\char232}}1
  {й}{{\selectfont\char233}}1
  {к}{{\selectfont\char234}}1
  {л}{{\selectfont\char235}}1
  {м}{{\selectfont\char236}}1
  {н}{{\selectfont\char237}}1
  {о}{{\selectfont\char238}}1
  {п}{{\selectfont\char239}}1
  {р}{{\selectfont\char240}}1
  {с}{{\selectfont\char241}}1
  {т}{{\selectfont\char242}}1
  {у}{{\selectfont\char243}}1
  {ф}{{\selectfont\char244}}1
  {х}{{\selectfont\char245}}1
  {ц}{{\selectfont\char246}}1
  {ч}{{\selectfont\char247}}1
  {ш}{{\selectfont\char248}}1
  {щ}{{\selectfont\char249}}1
  {ъ}{{\selectfont\char250}}1
  {ы}{{\selectfont\char251}}1
  {ь}{{\selectfont\char252}}1
  {э}{{\selectfont\char253}}1
  {ю}{{\selectfont\char254}}1
  {я}{{\selectfont\char255}}1
  {А}{{\selectfont\char192}}1
  {Б}{{\selectfont\char193}}1
  {В}{{\selectfont\char194}}1
  {Г}{{\selectfont\char195}}1
  {Д}{{\selectfont\char196}}1
  {Е}{{\selectfont\char197}}1
  {Ё}{{\"E}}1
  {Ж}{{\selectfont\char198}}1
  {З}{{\selectfont\char199}}1
  {И}{{\selectfont\char200}}1
  {Й}{{\selectfont\char201}}1
  {К}{{\selectfont\char202}}1
  {Л}{{\selectfont\char203}}1
  {М}{{\selectfont\char204}}1
  {Н}{{\selectfont\char205}}1
  {О}{{\selectfont\char206}}1
  {П}{{\selectfont\char207}}1
  {Р}{{\selectfont\char208}}1
  {С}{{\selectfont\char209}}1
  {Т}{{\selectfont\char210}}1
  {У}{{\selectfont\char211}}1
  {Ф}{{\selectfont\char212}}1
  {Х}{{\selectfont\char213}}1
  {Ц}{{\selectfont\char214}}1
  {Ч}{{\selectfont\char215}}1
  {Ш}{{\selectfont\char216}}1
  {Щ}{{\selectfont\char217}}1
  {Ъ}{{\selectfont\char218}}1
  {Ы}{{\selectfont\char219}}1
  {Ь}{{\selectfont\char220}}1
  {Э}{{\selectfont\char221}}1
  {Ю}{{\selectfont\char222}}1
  {Я}{{\selectfont\char223}}1
  {і}{{\selectfont\char105}}1
  {ї}{{\selectfont\char168}}1
  {є}{{\selectfont\char185}}1
  {ґ}{{\selectfont\char160}}1
  {І}{{\selectfont\char73}}1
  {Ї}{{\selectfont\char136}}1
  {Є}{{\selectfont\char153}}1
  {Ґ}{{\selectfont\char128}}1
  {\{}{{{\color{brackets}\{}}}1 % Цвет скобок {
  {\}}{{{\color{brackets}\}}}}1 % Цвет скобок }
}
 
 % Для настройки заголовка кода
\usepackage{caption}
\DeclareCaptionFont{white}{\color{сaptiontext}}
\DeclareCaptionFormat{listing}{\parbox{\linewidth}{\colorbox{сaptionbk}{\parbox{\linewidth}{#1#2#3}}\vskip-4pt}}
\captionsetup[lstlisting]{format=listing,labelfont=white,textfont=white}
\renewcommand{\lstlistingname}{Код} % Переименование Listings в нужное именование структуры


%------------------------------------------------------------------------------

\author{В.~В.~Дьячков}
\title{Программирование}
\begin{document}
\listoftodos
\maketitle
\chapter{Основные конструкции языка}
%############################################################
\section{Задание 1}
\subsection{Задание}

Перевести длину отрезка из дюймов в метры, сантиметры и миллиметры.

\subsection{Теоретические сведения}
\hspace{\parindent}Для реализации данной задачи я воспользовался структурой \texttt{struct} для целостного представления трех величин: метров, сантиметров и миллиметров. 

Так же были использованы стандартные функции ввода-вывода \texttt{scanf}, \texttt{printf}, \texttt{puts} из библиотеки \textit{stdio.h}.

При помощи оператоов ветвления \texttt{if-else} и \texttt{switch} реализовано интерактивное подменю для более удобного взаимодествия пользователя с программой.

Для решения поставленной задачи воспользовался математическим фактом: 1 дюйм = 2.54 см.

\subsection{Проектирование}
\hspace{\parindent}В ходе проектирования решено выделить 5 функций:
\begin{enumerate}
	\item \texttt{calculating\_inch\_to\_cm} ---
	в этой функции реализован перевод дюймов в метры, сантиметры и миллиметры.
	\\В качестве передаваемых параметров используются целочисленное \texttt{int} значение дюймов и объявленная в заголовочном файле \textit{inch\_to\_cm.h} структура \texttt{Meters}, передаваемая по ссылке.
		 
	\item \texttt{menu\_inch\_to\_cm} ---
	в этой функции реализованно начальное взаимодействие с пользователем. Пользователю предлагается выбрать консольный ввод, вызов справки, возварт к главное меню или завершение программы.	
		 
	\item \texttt{input\_inch\_to\_cm} ---
	в этой функции реализованно основное взаимодействие с пользователем. Пользователю предлагается ввести дюймы, после чего вызывается функция для перевода в метрическую систему и функция для вывода получившегося результата в консоль.
	
	\item \texttt{show\_inch\_to\_cm} ---
	в этой функции происходит вывод в консоль получившегося результата. Для этого в качестве параметров передаются целочисленное \texttt{int} значение дюймов, заданное пользователем, и структура \texttt{Meters}, в которой находятся результаты наших вычислений.
	
	\item \texttt{help\_inch\_to\_cm} ---
	в этой функции происходит вывод в консоль формулировки задания, помагающая пользователю в использовании программы.
\end{enumerate}

\subsection{Описание тестового стенда и методики тестирования}
\hspace{\parindent}Среда разработки QtCreator 3.5.1, компилятор GCC 4.8.4 (x86 64 bit), операционная система Linux Mint 17.2 Cinnamon 64 bit.

Автоматическое модульное тестирование реализовано при помощи QtTest (Приложение 1).

\subsection{Тестовый план и результаты тестирования}
\hspace{\parindent}При заданном занчении \texttt{inches = 301} вызывается функция перевода в метрическую систему \texttt{calculating\_inch\_to\_cm}, после чего происходит сравнение ожидаемых и реальных результатов при помощи макроса \texttt{QCOMPARE}. Тест пройден: 
\begin{center}301 дюйм = 7 метров 64 сантиметра 5.4 миллиметра.\end{center}

\subsection{Выводы}
\hspace{\parindent}При выполнении задания я закрепил свои навыки в работе с основными конструкциями языка \texttt{C} и получил опыт в организации многофайлового проекта и создании модульных тестов.

\newpage
\subsection*{Листинги}
inch\_to\_cm.h
\lstinputlisting[]
{../sources/homework/library/inch_to_cm.h}
inch\_to\_cm.c
\lstinputlisting[]
{../sources/homework/library/inch_to_cm.c}
ui\_inch\_to\_cm.h
\lstinputlisting[]
{../sources/homework/app/ui_inch_to_cm.h}
ui\_inch\_to\_cm.c
\lstinputlisting[]
{../sources/homework/app/ui_inch_to_cm.c}

\newpage
%############################################################

\section{Задание 2}
\subsection{Задание}
Определить, за какое время путник одолел первую половину пути, двигаясь T1 часов со скоростью V1, T2 часов со скоростью V2, T3 часов со скоростью V3.

\subsection{Теоритические сведения}
\hspace{\parindent}Для реализации данной задачи воспользовался макросом \texttt{\#define NUMBER\_OF\_PIECES 3} для того, чтобы абстрагироваться от количества частей пути, на которых путник двигался с различной скоростью. Благодаря такому приему с помощью лишь одной замены в заголовочном файле мы можем изменить количество таких участков, не потеряв работоспособность программы.

Так же были использованы стандартные функции ввода-вывода \texttt{scanf}, \texttt{printf}, \texttt{puts} из библиотеки \textit{stdio.h}.

При помощи оператоов ветвления \texttt{if-else} и \texttt{switch} реализовано интерактивное подменю для более удобного взаимодествия пользователя с программой. С использованием оператора цикла \texttt{for} происходит иттерирование по каждому участку пути.

Для решения поставленной задачи я воспользовался математическими фактами: 
\begin{itemize}
\item чтобы найти путь на отдельном участке пути необхадимо умножить скорость на даном участке на время, затраченное на прохождение этого участка;
\item чтобы найти общий путь необходимо сложить пути всех участков.
\end{itemize}

\subsection{Проектирование}
\hspace{\parindent}В ходе проектирования решено выделить 6 функций:
\begin{enumerate}
	\item \texttt{halfdistance\_time} ---
	в этой функции реализовано нахождение половины пройденного пути.
	\\В качестве передаваемых параметров используются 2 указателя на тип \texttt{double} -- \textbf{скорости}  на участках пути и \textbf{время}, затраченное на прохождение каждого участка пути.
		 
	\item \texttt{calculating\_time} ---
	в этой функции реализованно вычисление времени, затраченного на прохождение первой половины пути. 
	\\В качестве передаваемых параметров используются 2 указателя на тип \texttt{double} -- \textbf{скорости}  на участках пути и \textbf{время}, затраченное на прохождение каждого участка пути.	
		 
	\item \texttt{menu\_time} ---
	в этой функции реализованно начальное взаимодействие с пользователем. Пользователю предлагается выбрать консольный ввод, вызов справки, возварт к главное меню или завершение программы.
	
	\item \texttt{input\_time} ---
	в этой функции реализованно основное взаимодействие с пользователем. Пользователю предлагается последовательно ввести скорость и время для каждого участка пути, после чего вызывается функция для вычисления времени, затраченного на половину пути и функция для вывода получившегося результата в консоль.
	
	\item \texttt{show\_time} ---
	в этой функции происходит вывод в консоль получившегося результата. Для этого в качестве параметров передаются вещественное число \texttt{double} -- вычситанное значение времени.

	\item \texttt{help\_time} ---
	в этой функции происходит вывод в консоль формулировки задания, помагающая пользователю в использовании программы.
\end{enumerate}

\subsection{Описание тестового стенда и методики тестирования}
\hspace{\parindent}Среда разработки QtCreator 3.5.1, компилятор GCC 4.8.4 (x86 64 bit), операционная система Linux Mint 17.2 Cinnamon 64 bit.

Автоматическое модульное тестирование реализовано при помощи QtTest (Приложение 1).

\subsection{Тестовый план и результаты тестирования}
\hspace{\parindent}При заданных значениях \mbox{\texttt{velocity[3] = {50, 100, 150;}}} и \\\mbox{\texttt{time[3] = {1, 1, 1;}}} вызывается функция вычисления времени, затраченного на прохождение первой половины пути \texttt{calculating\_time}, после чего происходит сравнение ожидаемых и реальных результатов при помощи макроса \texttt{QCOMPARE}. Тест пройден: 
\begin{center}Искомое время = 2 часа.\end{center}

\subsection{Выводы}
\hspace{\parindent}При выполнении задания я закрепил свои навыки в работе с основными конструкциями языка \texttt{C} и получил опыт в организации многофайлового проекта и создании модульных тестов.

\newpage
\subsection*{Листинги}
time.h
\lstinputlisting[]
{../sources/homework/library/time.h}
time.c
\lstinputlisting[]
{../sources/homework/library/time.c}
ui\_time.h
\lstinputlisting[]
{../sources/homework/app/ui_time.h}
ui\_time.c
\lstinputlisting[]
{../sources/homework/app/ui_time.c}
\newpage
%############################################################

\chapter{Циклы}
\section{Задание 1}
\subsection{Задание}
\subsection{Теоритические сведения}
\subsection{Проектирование}
\subsection{Описание тестового стенда и методики тестирования}
\subsection{Тестовый план и результаты тестирования}
\subsection{Выводы}

\end{document}