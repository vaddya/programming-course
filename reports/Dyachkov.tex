\documentclass[12pt,a4paper]{report}
\usepackage{ucs} 
\usepackage[utf8]{inputenc}
\usepackage[russian]{babel}
\usepackage[OT1]{fontenc}
\usepackage{amsmath}
\usepackage{amsfonts}
\usepackage{amssymb}
\usepackage{graphicx}
\usepackage{cmap}					% поиск в PDF
\usepackage{mathtext} 				% русские буквы в формулах
%\usepackage{tikz-uml}               % uml диаграммы

% TODOs
\usepackage[%
  colorinlistoftodos,
  shadow
]{todonotes}

% Генератор текста
\usepackage{blindtext}

\usepackage[section]{placeins}

%------------------------------------------------------------------------------

% Подсветка синтаксиса
\usepackage{color}
\usepackage{xcolor}
\usepackage{listings}
 
 % Цвета для кода
\definecolor{string}{HTML}{B40000} % цвет строк в коде
\definecolor{comment}{HTML}{008000} % цвет комментариев в коде
\definecolor{keyword}{HTML}{1A00FF} % цвет ключевых слов в коде
\definecolor{morecomment}{HTML}{8000FF} % цвет include и других элементов в коде
\definecolor{captiontext}{HTML}{FFFFFF} % цвет текста заголовка в коде
\definecolor{captionbk}{HTML}{999999} % цвет фона заголовка в коде
\definecolor{bk}{HTML}{FFFFFF} % цвет фона в коде
\definecolor{frame}{HTML}{999999} % цвет рамки в коде
\definecolor{brackets}{HTML}{B40000} % цвет скобок в коде
 
 % Настройки отображения кода
\lstset{
language=C, % Язык кода по умолчанию
morekeywords={*,...}, % если хотите добавить ключевые слова, то добавляйте
 % Цвета
keywordstyle=\color{keyword}\ttfamily\bfseries,
stringstyle=\color{string}\ttfamily,
commentstyle=\color{comment}\ttfamily\itshape,
morecomment=[l][\color{morecomment}]{\#}, 
 % Настройки отображения     
breaklines=true, % Перенос длинных строк
basicstyle=\ttfamily\footnotesize, % Шрифт для отображения кода
backgroundcolor=\color{bk}, % Цвет фона кода
%frame=lrb,xleftmargin=\fboxsep,xrightmargin=-\fboxsep, % Рамка, подогнанная к заголовку
frame=tblr
rulecolor=\color{frame}, % Цвет рамки
tabsize=3, % Размер табуляции в пробелах
showstringspaces=false,
 % Настройка отображения номеров строк. Если не нужно, то удалите весь блок
numbers=left, % Слева отображаются номера строк
stepnumber=1, % Каждую строку нумеровать
numbersep=5pt, % Отступ от кода 
numberstyle=\small\color{black}, % Стиль написания номеров строк
 % Для отображения русского языка
extendedchars=true,
literate={Ö}{{\"O}}1
  {Ä}{{\"A}}1
  {Ü}{{\"U}}1
  {ß}{{\ss}}1
  {ü}{{\"u}}1
  {ä}{{\"a}}1
  {ö}{{\"o}}1
  {~}{{\textasciitilde}}1
  {а}{{\selectfont\char224}}1
  {б}{{\selectfont\char225}}1
  {в}{{\selectfont\char226}}1
  {г}{{\selectfont\char227}}1
  {д}{{\selectfont\char228}}1
  {е}{{\selectfont\char229}}1
  {ё}{{\"e}}1
  {ж}{{\selectfont\char230}}1
  {з}{{\selectfont\char231}}1
  {и}{{\selectfont\char232}}1
  {й}{{\selectfont\char233}}1
  {к}{{\selectfont\char234}}1
  {л}{{\selectfont\char235}}1
  {м}{{\selectfont\char236}}1
  {н}{{\selectfont\char237}}1
  {о}{{\selectfont\char238}}1
  {п}{{\selectfont\char239}}1
  {р}{{\selectfont\char240}}1
  {с}{{\selectfont\char241}}1
  {т}{{\selectfont\char242}}1
  {у}{{\selectfont\char243}}1
  {ф}{{\selectfont\char244}}1
  {х}{{\selectfont\char245}}1
  {ц}{{\selectfont\char246}}1
  {ч}{{\selectfont\char247}}1
  {ш}{{\selectfont\char248}}1
  {щ}{{\selectfont\char249}}1
  {ъ}{{\selectfont\char250}}1
  {ы}{{\selectfont\char251}}1
  {ь}{{\selectfont\char252}}1
  {э}{{\selectfont\char253}}1
  {ю}{{\selectfont\char254}}1
  {я}{{\selectfont\char255}}1
  {А}{{\selectfont\char192}}1
  {Б}{{\selectfont\char193}}1
  {В}{{\selectfont\char194}}1
  {Г}{{\selectfont\char195}}1
  {Д}{{\selectfont\char196}}1
  {Е}{{\selectfont\char197}}1
  {Ё}{{\"E}}1
  {Ж}{{\selectfont\char198}}1
  {З}{{\selectfont\char199}}1
  {И}{{\selectfont\char200}}1
  {Й}{{\selectfont\char201}}1
  {К}{{\selectfont\char202}}1
  {Л}{{\selectfont\char203}}1
  {М}{{\selectfont\char204}}1
  {Н}{{\selectfont\char205}}1
  {О}{{\selectfont\char206}}1
  {П}{{\selectfont\char207}}1
  {Р}{{\selectfont\char208}}1
  {С}{{\selectfont\char209}}1
  {Т}{{\selectfont\char210}}1
  {У}{{\selectfont\char211}}1
  {Ф}{{\selectfont\char212}}1
  {Х}{{\selectfont\char213}}1
  {Ц}{{\selectfont\char214}}1
  {Ч}{{\selectfont\char215}}1
  {Ш}{{\selectfont\char216}}1
  {Щ}{{\selectfont\char217}}1
  {Ъ}{{\selectfont\char218}}1
  {Ы}{{\selectfont\char219}}1
  {Ь}{{\selectfont\char220}}1
  {Э}{{\selectfont\char221}}1
  {Ю}{{\selectfont\char222}}1
  {Я}{{\selectfont\char223}}1
  {і}{{\selectfont\char105}}1
  {ї}{{\selectfont\char168}}1
  {є}{{\selectfont\char185}}1
  {ґ}{{\selectfont\char160}}1
  {І}{{\selectfont\char73}}1
  {Ї}{{\selectfont\char136}}1
  {Є}{{\selectfont\char153}}1
  {Ґ}{{\selectfont\char128}}1
  {\{}{{{\color{brackets}\{}}}1 % Цвет скобок {
  {\}}{{{\color{brackets}\}}}}1 % Цвет скобок }
}
 
 % Для настройки заголовка кода
\usepackage{caption}
\DeclareCaptionFont{white}{\color{сaptiontext}}
\DeclareCaptionFormat{listing}{\parbox{\linewidth}{\colorbox{сaptionbk}{\parbox{\linewidth}{#1#2#3}}\vskip-4pt}}
\captionsetup[lstlisting]{format=listing,labelfont=white,textfont=white}
\renewcommand{\lstlistingname}{Код} % Переименование Listings в нужное именование структуры


%------------------------------------------------------------------------------

\author{В.~В.~Дьячков}
\title{Программирование}
\begin{document}
%\listoftodos
\maketitle

\tableofcontents{}

\chapter{Основные конструкции языка}
%############################################################
\section{Задание 1.1. Перевод дюймов в метрическую систему}
\subsection{Задание}

\hspace{\parindent}Перевести длину отрезка из дюймов в метры, сантиметры и миллиметры.

\subsection{Теоретические сведения}
\hspace{\parindent}Для реализации данной задачи была использована структура \texttt{struct} для удобства представления трех величин: метров, сантиметров и миллиметров. 

Так же были использованы стандартные функции ввода-вывода \texttt{scanf}, \texttt{printf}, \texttt{puts} из стандартной библиотеки языка \verb+С+, объявленные в заголовочном файле \textit{stdio.h}.

При помощи операторов ветвления \texttt{if-else} и \texttt{switch} реализовано интерактивное подменю для более удобного взаимодествия пользователя с программой.

Для решения поставленной задачи воспользовался метричский фактом: 1 дюйм = 2.54 см.

\subsection{Проектирование}
\hspace{\parindent}В ходе проектирования было решено выделить 5 функций:
\begin{enumerate}
 	\item Перевод дюймов в метры, сантиметры и миллиметры
 	
 	\verb+void calculating_inch_to_cm(int, Meters *);+
 	
	Параметрами функции являются целочисленное \verb+int+ значение дюймов и указатель на структуру \verb+Meters+, в которой будут содержаться значения для метров, сантиметров, миллиметров в результате работы функции.
Структура объявлена в заголовочном файле \verb+inch_to_cm.h+		 
		 
		 
	\item Меню с первоначальным пользовательским взаимодействием
	
	\verb+void menu_inch_to_cm();+
	
	Пользователю предлагается выбрать консольный ввод, вызов справки, возварт к главное меню или завершение программы.	
		 
		 
	\item Основное пользовательское взаимодействие
	
	\verb+void input_inch_to_cm();+

	Пользователю предлагается ввести дюймы, после чего вызывается функция для перевода в метрическую систему и функция для вывода получившегося результата в консоль.
	
	
	\item Вывод в консоль получившегося результата
	
	\verb+void show_inch_to_cm(int, Meters);+

	Параметрами функции являются целочисленное \verb+int+ значение дюймов заданное пользователем, и структура \verb+Meters+, в которой содержаться значения для метров, сантиметров, миллиметров в результате работы функции.
Структура объявлена в заголовочном файле \verb+inch_to_cm.h+	
	
	
	\item Вспомогательная информация
	
	\verb+void help_inch_to_cm();+
	
	Вывод в консоль формулировки задания, помагающая пользователю в использовании программы.
\end{enumerate}

\subsection{Описание тестового стенда и методики тестирования}
\hspace{\parindent}Среда разработки QtCreator 3.5.1, компилятор GCC 4.8.4 (x86 64 bit), операционная система Linux Mint 17.2 Cinnamon 64 bit.
В процессе выполнения задания производилось ручное тестирование.
Модульное тестирование реализовано при помощи фреймворка QtTest.

\subsection{Тестовый план и результаты тестирования}
\hspace{\parindent}В таблице \ref{inch_to_cm_test_plan} представлены значения дюймов использованные при тестировании и ожидаемые значения для метров, сантиметров и миллиметров, а также отметка о результате теста.
\FloatBarrier
\begin{table}[h]
\caption{Тестовый план и результаты тестирования перевода дюймов в метрическую систему}
\label{inch_to_cm_test_plan}
\begin{tabular}{|c|c c c|c|c|}
\hline 
Дюймы & Метры & Сантиметры & Миллиметры & Тип теста & Результат \\ 
\hline 
301 & 7 & 64 & 5.4 & Модульный & Успешно \\ 
\hline 
100 & 2 & 54 & 0 & Ручной & Успешно \\ 
\hline 
\end{tabular} 
\end{table}
\FloatBarrier
Все тесты пройдены успешно. Листинги модульных тестов приведены в приложении \ref{unit_tests_1_4}.
\subsection{Выводы}
\hspace{\parindent}При выполнении задания были закреплены навыки в работе с основными конструкциями языка \texttt{C} и получен опыт в организации многофайлового проекта и создании модульных тестов.

\newpage
\subsection{Листинги}
\verb+inch_to_cm.h+
\lstinputlisting[]
{../sources/homework/library/inch_to_cm.h}
\verb+inch_to_cm.c+
\lstinputlisting[]
{../sources/homework/library/inch_to_cm.c}
\verb+ui_inch_to_cm.h+
\lstinputlisting[]
{../sources/homework/app/ui_inch_to_cm.h}
\verb+ui_inch_to_cm.c+
\lstinputlisting[]
{../sources/homework/app/ui_inch_to_cm.c}




\newpage
%############################################################




\section{Задание 1.2. Расчет времени}
\subsection{Задание}
\hspace{\parindent}Определить, за какое время путник одолел первую половину пути, двигаясь T1 часов со скоростью V1, T2 часов со скоростью V2, T3 часов со скоростью V3.

\subsection{Теоритические сведения}
\hspace{\parindent}Для того, чтобы абстрагироваться от количества частей пути, на которых путник двигался с различной скоростью, в реализации задачи был использован макрос \verb+#define NUMBER_OF_PIECES 3+. Благодаря такому приему с помощью лишь одной замены в заголовочном файле \textit{time.h} мы можем изменить количество таких участков, не потеряв работоспособность программы.

Так же были использованы стандартные функции ввода-вывода \texttt{scanf}, \texttt{printf}, \texttt{puts} из стандартной библиотеки языка \verb+С+, объявленные в заголовочном файле \textit{stdio.h}.

При помощи оператоов ветвления \texttt{if-else} и \texttt{switch} реализовано интерактивное подменю для более удобного взаимодествия пользователя с программой. С использованием оператора цикла \texttt{for} происходит иттерирование по каждому участку пути.

Для решения поставленной задачи были использованы следующие математические факты: 
\begin{itemize}
\item чтобы найти путь на отдельном участке пути необхадимо умножить скорость на даном участке на время, затраченное на прохождение этого участка;
\item чтобы найти общий путь необходимо сложить пути всех участков.
\end{itemize}

\subsection{Проектирование}
\hspace{\parindent}В ходе проектирования было решено выделить 6 функций:
\begin{enumerate}
	\item Нахождение половины пройденного пути
	
	\verb+double halfdistance_time(double *, double *);+
	
	В качестве передаваемых параметров используются 2 указателя на тип \texttt{double *} -- \textbf{скорости}  на участках пути и \textbf{время}, затраченное на прохождение каждого такого участка. Возвращаемым значением является расстояние типа \verb+double+, равное половине пройденного пути.
		 
	\item Вычисление времени, затраченного на прохождение первой половины пути
	
	\verb+double calculating_time(double *, double *);+
	
	В качестве передаваемых параметров используются 2 указателя на тип \texttt{double *} -- \textbf{скорости}  на участках пути и \textbf{время}, затраченное на прохождение каждого такого учатска. Возвращаемым значением является рассчитанное время типа \verb+double+, затраченное на прохождение половины пути.
		 
	\item Меню с начальным пользовательским взаимодействием
	
	\verb+void menu_time();+
	
	Пользователю предлагается выбрать консольный ввод, вызов справки, возварт к главное меню или завершение программы.
	
	\item Основное пользовательское взаимодействие
	
	\verb+void input_time();+
	
	Пользователю предлагается последовательно ввести скорость и время для каждого участка пути, после чего вызывается функция для вычисления времени, затраченного на половину пути и функция для вывода получившегося результата в консоль.
	
	\item Вывод в консоль получившегося результата
	
	\verb+void show_time(double);+
	
	Для этого в качестве параметров передаются вещественное число \texttt{double} -- вычситанное значение времени.

	\item Вспомогательная информация
	
	\verb+void help_time();+
	
	Вывод в консоль формулировки задания, помагающая пользователю в использовании программы.
\end{enumerate}


\subsection{Описание тестового стенда и методики тестирования}
\hspace{\parindent}Среда разработки QtCreator 3.5.1, компилятор GCC 4.8.4 (x86 64 bit), операционная система Linux Mint 17.2 Cinnamon 64 bit.
В процессе выполнения задания производилось ручное тестирование.
Модульное тестирование реализовано при помощи фреймворка QtTest.

\subsection{Тестовый план и результаты тестирования}
\hspace{\parindent}В таблице \ref{time_test_plan} представлены значения дюймов использованные при тестировании и ожидаемые значения для метров, сантиметров и миллиметров, а также отметка о результате теста.
\FloatBarrier
\begin{table}[h]
\caption{Тестовый план и результаты тестирования расчета времени, затраченнего на половину пути}
\label{time_test_plan}
\begin{tabular}{| c c c | c c c | c | c | c |}
\hline 
V1 & V2 & V3 & T1 & T2 & T3 & Результат & Тип теста & Результат \\ 
\hline 
60 & 80 & 100 & 4 & 2 & 5 & 6.5 & Модульный & Успешно \\ 
\hline 
50 & 100 & 150 & 1 & 1 & 1 & 2.0 & Ручной & Успешно \\ 
\hline 
\end{tabular} 
\end{table}
\FloatBarrier
Все тесты пройдены успешно. Листинги модульных тестов приведены в приложении \ref{unit_tests_1_4}.
\subsection{Выводы}
\hspace{\parindent}При выполнении задания закреплены навыки в работе с основными конструкциями языка \verb+C+ и получен опыт в организации многофайлового проекта и создании модульных тестов.

\newpage
\subsection{Листинги}
\subsubsection*{time.h}
\lstinputlisting[]
{../sources/homework/library/time.h}
time.c
\lstinputlisting[]
{../sources/homework/library/time.c}
ui\_time.h
\lstinputlisting[]
{../sources/homework/app/ui_time.h}
ui\_time.c
\lstinputlisting[]
{../sources/homework/app/ui_time.c}



\newpage
%############################################################



\chapter{Циклы}
\section{Задание 2. Палиндром}
\subsection{Задание}

\hspace{\parindent}Проверить, является ли заданное число палиндромом.

\subsection{Теоретические сведения}
\hspace{\parindent}Для реализации данной задачи были использованы стандартные функции ввода-вывода \texttt{scanf}, \texttt{printf}, \texttt{puts} из стандартной библиотеки языка \verb+С+, объявленные в заголовочном файле \textit{stdio.h}.

При помощи операторов ветвления \texttt{if-else} и \texttt{switch} реализовано интерактивное подменю для более удобного взаимодествия пользователя с программой.

Для решения поставленной задачи использовался факт: число называется палиндромом, если первая цифра равна последней, вторая предпоследней и так далее.

\subsection{Проектирование}
\hspace{\parindent}В ходе проектирования было решено выделить 5 функций:1
\begin{enumerate}
 	\item Проверка числа на палиндром
 	
 	\verb+int is_palindrome(char *);+
 	
	Параметром функции является указатель на строку \verb+char *+, содержащую число, представленное в виде символов \verb+char+. Возвращаемое значение \verb+int+ равно \verb+1+, если число является палиндромом и \verb+0+, если оно таковым не является.	 
		 
		 
	\item Меню с первоначальным пользовательским взаимодействием
	
	\verb+void menu_palindrome();+
	
	Пользователю предлагается выбрать консольный ввод, вызов справки, возварт к главное меню или завершение программы.	
		 
		 
	\item Основное пользовательское взаимодействие
	
	\verb+void input_palindrome();+

	Пользователю предлагается ввести число, после чего вызывается функция для опредления палиндрома и функция для вывода получившегося результата в консоль.
	
	
	\item Вывод в консоль получившегося результата
	
	\verb+void show_palindrome(char *, int);+

Параметрами функции являеются указатель на строку \verb+char *+ -- число, которое проверяется на палиндром, и целочисленное значение \verb+int+ равное \verb+1+, если число является палиндромом и \verb+0+, если оно таковым не является.	 
	
	
	\item Вспомогательная информация
	
	\verb+void help_palindrome();+
	
	Вывод в консоль формулировки задания, помагающая пользователю в использовании программы.
\end{enumerate}

\subsection{Описание тестового стенда и методики тестирования}
\hspace{\parindent}Среда разработки QtCreator 3.5.1, компилятор GCC 4.8.4 (x86 64 bit), операционная система Linux Mint 17.2 Cinnamon 64 bit.
В процессе выполнения задания производилось ручное тестирование.
Модульное тестирование реализовано при помощи фреймворка QtTest.

\subsection{Тестовый план и результаты тестирования}
\hspace{\parindent}В таблице \ref{inch_to_cm_test_plan} представлены значения дюймов использованные при тестировании и ожидаемые значения для метров, сантиметров и миллиметров, а также отметка о результате теста.
\FloatBarrier
\begin{table}[h]
\caption{Тестовый план и результаты тестирования перевода дюймов в метрическую систему}
\label{inch_to_cm_test_plan}
\begin{tabular}{| c | c | c | c |}
\hline 
Число & Палиндром & Тип теста & Результат\\ 
\hline 
2311441132 & Да & Модульный & Успешно \\ 
\hline 
525321 & Нет & Модульный & Успешно \\ 
\hline 
123464321 & Да & Ручной & Успешно \\ 
\hline 
165332 & Нет & Ручной & Успешно \\ 
\hline 
\end{tabular} 
\end{table}
\FloatBarrier
Все тесты пройдены успешно. Листинги модульных тестов приведены в приложении \ref{unit_tests_1_4}.
\subsection{Выводы}
\hspace{\parindent}При выполнении задания были закреплены навыки в работе с оператором цикла \verb+for+ и получен опыт в организации многофайлового проекта и создании модульных тестов.

\newpage
\subsection{Листинги}
\verb+palindrome.h+
\lstinputlisting[]
{../sources/homework/library/palindrome.h}
\verb+palindrome.c+
\lstinputlisting[]
{../sources/homework/library/palindrome.c}
\verb+ui_palindrome.h+
\lstinputlisting[]
{../sources/homework/app/ui_palindrome.h}
\verb+ui_palindrome.c+
\lstinputlisting[]
{../sources/homework/app/ui_palindrome.c}


\newpage
%############################################################



\chapter{Массивы}
\section{Задание 3. Игра}
\subsection{Задание}

\hspace{\parindent}По кругу располагаются n человек. Ведущий считает по кругу, начиная с первого, и выводит m-ого человека. Круг смыкается, счет возобновляется со следующего; так  продолжается, пока в круге не останется только один человек. Найти номер этого человека.

\subsection{Теоретические сведения}
\hspace{\parindent}Для реализации данной задачи были использованы ряд стандартных функций ввода-вывода, таких как \texttt{scanf}, \texttt{printf}, \texttt{puts}, \texttt{fopen}, \texttt{fclose}, \texttt{fscanf}, \texttt{fprintf} из стандартной библиотеки языка \verb+С+, объявленные в заголовочном файле \textit{stdio.h}, а так же функции для работы с памятью \verb+malloc+ и \verb+free+, объявленные в заголовочном файле \textit{stdlib.h}.

При помощи операторов ветвления \texttt{if-else} и \texttt{switch} реализовано интерактивное подменю для более удобного взаимодествия пользователя с программой. При помощи цикла \verb+for+ и \verb+while+ происходит обращение к эллементам массива.

Для решения поставленной задачи использовался разработанный алгоритм для определения искомого номера, основанный на иттериовании по массиву.

\subsection{Проектирование}
\hspace{\parindent}В ходе проектирования было решено выделить 6 функций:
\begin{enumerate}
 	\item Определение номера победителя
 	
 	\verb+int determine_the_winner(int, int);+
 	
	Параметром функции является два целочисленных значения \verb+int+ -- количество игроков и номер, игрока с которым выводят из круга. Возвращаемое значение \verb+int+ равно номеру победившего игрока.	 
		 
		 
	\item Меню с первоначальным пользовательским взаимодействием
	
	\verb+void menu_circle_game();+
	
	Пользователю предлагается выбрать консольный ввод, файловый ввод, вызов справки, возварт к главное меню или завершение программы. 	 
		 
	\item Консольное взаимодействие с пользователем
	
	\verb+void input_console_circle_game();+

	Пользователю предлагается ввести число, после чего вызывается функция для опредления победителя и функция для вывода получившегося результата в консоль.
	
	\item Файловое взаимодействие с пользователем
	
	\verb+void input_file_circle_game();+

	Пользователю предлагается ввести имя файла, откуда программа должна получить исходные данные, и имя файла, куда будет происходить печать результата, после чего вызывается функция для опредления победителя и вывод результата в указанный файл.
	
	\item Вывод в консоль получившегося результата
	
	\verb+void show_circle_game(int);+

Параметром функции является и целочисленное значение \verb+int+ равное номеру победившего игрока.
	
	
	\item Вспомогательная информация
	
	\verb+void help_circle_game();+
	
	Вывод в консоль формулировки задания, помагающая пользователю в использовании программы.
\end{enumerate}

\subsection{Описание тестового стенда и методики тестирования}
\hspace{\parindent}Среда разработки QtCreator 3.5.1, компилятор GCC 4.8.4 (x86 64 bit), операционная система Linux Mint 17.2 Cinnamon 64 bit.
В процессе выполнения задания производилось ручное тестирование.
Модульное тестирование реализовано при помощи фреймворка QtTest.

\subsection{Тестовый план и результаты тестирования}
\hspace{\parindent}В таблице \ref{inch_to_cm_test_plan} представлены значения дюймов использованные при тестировании и ожидаемые значения для метров, сантиметров и миллиметров, а также отметка о результате теста.
\FloatBarrier
\begin{table}[h]
\caption{Тестовый план и результаты тестирования перевода дюймов в метрическую систему}
\label{inch_to_cm_test_plan}
\begin{tabular}{| c c | c | c | c |}
\hline 
Игроков & Выводят номер & Победитель & Тип теста & Результат\\ 
\hline 
32 & 5 & 13 & Модульный & Успешно \\ 
\hline 
9 & 45 & 7 & Модульный & Успешно \\ 
\hline 
5 & 3 & 4 & Ручной & Успешно \\ 
\hline 
\end{tabular} 
\end{table}
\FloatBarrier
Все тесты пройдены успешно. Листинги модульных тестов приведены в приложении \ref{unit_tests_1_4}.
\subsection{Выводы}
\hspace{\parindent}При выполнении задания были закреплены навыки в работе с массивами и получен опыт в организации многофайлового проекта и создании модульных тестов.

\newpage
\subsection{Листинги}
\verb+circle_game.h+
\lstinputlisting[]
{../sources/homework/library/circle_game.h}
\verb+circle_game.c+
\lstinputlisting[]
{../sources/homework/library/circle_game.c}
\verb+ui_circle_game.h+
\lstinputlisting[]
{../sources/homework/app/ui_circle_game.h}
\verb+ui_circle_game.c+
\lstinputlisting[]
{../sources/homework/app/ui_circle_game.c}




\newpage
%############################################################



\chapter{Строки}
\section{Задание 4. Фразы}
\subsection{Задание}

\hspace{\parindent}Проверить, что все фразы в тексте  начинаются с прописной буквы и при необходимости откорректировать текст.

\subsection{Теоретические сведения}
\hspace{\parindent}Для реализации данной задачи были использованы ряд стандартных функций ввода-вывода, таких как \texttt{scanf}, \texttt{printf}, \texttt{puts}, \texttt{fopen}, \texttt{fclose}, \texttt{fscanf}, \texttt{fprintf} из стандартной библиотеки языка \verb+С+, объявленные в заголовочном файле \textit{stdio.h}, а так же функция для преобразования строчных букв в прописные \verb+toupper+, объявленная в заголовочном файле \textit{ctype.h} и функция \verb+strcpy+, объявленная в библиотеке \textit{string.h}.

При помощи операторов ветвления \texttt{if-else} и \texttt{switch} реализовано интерактивное подменю для более удобного взаимодествия пользователя с программой. При помощи цикла \verb+for+ и происходит обращение к символам строки.

Для решения поставленной задачи использовался правило русского языка: новое предложение должно начинаться с прописной буквы.

\subsection{Проектирование}
\hspace{\parindent}В ходе проектирования было решено выделить 6 функций:
\begin{enumerate}
 	\item Проверка и исправление переданной строки
 	
 	\verb+void upper_case_phrases(char *);+
 	
	Параметром функции является указтель на строку \verb+char *+, содержащий адрес переданной строки.	 
		 
		 
	\item Меню с первоначальным пользовательским взаимодействием
	
	\verb+void menu_phrases();+ 
	
	Пользователю предлагается выбрать консольный ввод, файловый ввод, вызов справки, возварт к главное меню или завершение программы. 	 
		 
	\item Консольное взаимодействие с пользователем
	
	\verb+void input_console_phrases();+

	Пользователю предлагается ввести строку, в которой необхадимо проверить правильность напасания фраз, после чего вызывается функция исправляющая ошибки и функция для вывода полученной строки в консоль.
	
	\item Файловое взаимодействие с пользователем
	
	\verb+void input_file_phrases();+

	Пользователю предлагается ввести имя файла, откуда программа должна получить исходную строку, и имя файла, куда будет происходить печать исправленной строки, после чего вызывается функция для исправления ошибок и вывод результата в указанный файл.
	
	\item Вывод в консоль получившегося результата
	
	\verb+void show_phrases(char *);+

Параметром функции является указтель на строку \verb+char *+, содержащий адрес исправленной строки.
	
	
	\item Вспомогательная информация
	
	\verb+void help_phrases();+
	
	Вывод в консоль формулировки задания, помагающая пользователю в использовании программы.
\end{enumerate}

\subsection{Описание тестового стенда и методики тестирования}
\hspace{\parindent}Среда разработки QtCreator 3.5.1, компилятор GCC 4.8.4 (x86 64 bit), операционная система Linux Mint 17.2 Cinnamon 64 bit.
В процессе выполнения задания производилось ручное тестирование.
Модульное тестирование реализовано при помощи фреймворка QtTest.

\subsection{Тестовый план и результаты тестирования}
\hspace{\parindent}В таблице \ref{inch_to_cm_test_plan} представлены значения дюймов использованные при тестировании и ожидаемые значения для метров, сантиметров и миллиметров, а также отметка о результате теста.
\FloatBarrier
\begin{table}[h]
\caption{Тестовый план и результаты тестирования перевода дюймов в метрическую систему}
\label{inch_to_cm_test_plan}
\begin{tabular}{| c | c | c | c |}
\hline 
Исходная строка & Исправленная строка & Тип теста & Результат\\ 
\hline 
check.check check.& Check.Check check. & Модульный & Успешно \\ 
\hline 
One. two three. four. & One. Two three. Four. & Ручной & Успешно \\ 
\hline 
\end{tabular} 
\end{table}
\FloatBarrier
Все тесты пройдены успешно. Листинги модульных тестов приведены в приложении \ref{unit_tests_1_4}.
\subsection{Выводы}
\hspace{\parindent}При выполнении задания были закреплены навыки в работе со строками и получен опыт в организации многофайлового проекта и создании модульных тестов.

\newpage
\subsection{Листинги}
\verb+phrases.h+
\lstinputlisting[]
{../sources/homework/library/phrases.h}
\verb+phrases.c+
\lstinputlisting[]
{../sources/homework/library/phrases.c}
\verb+ui_phrases.h+
\lstinputlisting[]
{../sources/homework/app/ui_phrases.h}
\verb+ui_phrases.c+
\lstinputlisting[]
{../sources/homework/app/ui_phrases.c}





\newpage
%############################################################



\chapter{Инкапсуляция}
\section{Задание 5. Массив}
\subsection{Задание}

\hspace{\parindent}Реализовать класс МАССИВ (целых чисел, переменного размера). Требуемые методы: конструктор, деструктор, копирование, индексация. 

\subsection{Теоретические сведения}
\hspace{\parindent}Для реализации данной задачи были использованы стандартные функций ввода-вывода языка \verb|C++|, таких как \verb+cin+ и \verb+cout+,  объявленные в заголовочном файле \textit{iostream},

При помощи цикла \verb+for+ и происходит иттерирование по массиву. С помощью операторов \verb+try+, \verb+throw+ и \verb+catch+ реализована обработка исключений. Работа с памятью происходит при помощи операторов \verb+new+ и \verb+delete+.

Для решения поставленной задачи использовался правило русского языка: новое предложение должно начинаться с прописной буквы.

\subsection{Проектирование}
\hspace{\parindent}В ходе проектирования было решено выделить 6 функций:
\begin{enumerate}
 	\item Проверка и исправление переданной строки
 	
 	\verb+void upper_case_phrases(char *);+
 	
	Параметром функции является указтель на строку \verb+char *+, содержащий адрес переданной строки.	 
		 
		 
	\item Меню с первоначальным пользовательским взаимодействием
	
	\verb+void menu_phrases();+ 
	
	Пользователю предлагается выбрать консольный ввод, файловый ввод, вызов справки, возварт к главное меню или завершение программы. 	 
		 
	\item Консольное взаимодействие с пользователем
	
	\verb+void input_console_phrases();+

	Пользователю предлагается ввести строку, в которой необхадимо проверить правильность напасания фраз, после чего вызывается функция исправляющая ошибки и функция для вывода полученной строки в консоль.
	
	\item Файловое взаимодействие с пользователем
	
	\verb+void input_file_phrases();+

	Пользователю предлагается ввести имя файла, откуда программа должна получить исходную строку, и имя файла, куда будет происходить печать исправленной строки, после чего вызывается функция для исправления ошибок и вывод результата в указанный файл.
	
	\item Вывод в консоль получившегося результата
	
	\verb+void show_phrases(char *);+

Параметром функции является указтель на строку \verb+char *+, содержащий адрес исправленной строки.
	
	
	\item Вспомогательная информация
	
	\verb+void help_phrases();+
	
	Вывод в консоль формулировки задания, помагающая пользователю в использовании программы.
\end{enumerate}

\subsection{Описание тестового стенда и методики тестирования}
\hspace{\parindent}Среда разработки QtCreator 3.5.1, компилятор GCC 4.8.4 (x86 64 bit), операционная система Linux Mint 17.2 Cinnamon 64 bit.
В процессе выполнения задания производилось ручное тестирование.
Модульное тестирование реализовано при помощи фреймворка QtTest.

\subsection{Тестовый план и результаты тестирования}
\hspace{\parindent}В таблице \ref{inch_to_cm_test_plan} представлены значения дюймов использованные при тестировании и ожидаемые значения для метров, сантиметров и миллиметров, а также отметка о результате теста.
\FloatBarrier
\begin{table}[h]
\caption{Тестовый план и результаты тестирования перевода дюймов в метрическую систему}
\label{inch_to_cm_test_plan}
\begin{tabular}{| c | c | c | c |}
\hline 
Исходная строка & Исправленная строка & Тип теста & Результат\\ 
\hline 
check.check check.& Check.Check check. & Модульный & Успешно \\ 
\hline 
One. two three. four. & One. Two three. Four. & Ручной & Успешно \\ 
\hline 
\end{tabular} 
\end{table}
\FloatBarrier
Все тесты пройдены успешно. Листинги модульных тестов приведены в приложении \ref{unit_tests_1_4}.
\subsection{Выводы}
\hspace{\parindent}При выполнении задания были закреплены навыки в работе со строками и получен опыт в организации многофайлового проекта и создании модульных тестов.

\newpage
\subsection{Листинги}
\verb+array.h+
\lstinputlisting[]
{../sources/homework/cpplib/array.h}
\verb+array.c+
\lstinputlisting[]
{../sources/homework/cpplib/array.cpp}
\verb+arrayapp.h+
\lstinputlisting[]
{../sources/homework/cpplib/arrayapp.h}
\verb+arrayapp.cpp+
\lstinputlisting[]
{../sources/homework/cpplib/arrayapp.cpp}




\chapter*{Приложения}
\subsection*{Листинги модульных тестов к заданиям с 1 по 4 включительно}
\lstinputlisting[label=unit_tests_1_4]
{../sources/homework/test/tst_testtest.cpp}
\end{document}